%We first give a brief overview of Neural Machine Translation before describing
%the model architecture of interest, the deep multi-layer recurrent model with
%LSTM. We then explain the different types of NMT weights
%together with our approaches to pruning and retraining.
%
%\subsection{Neural Machine Translation}
%Neural machine translation aims to directly model the conditional probability $p(\tgt{}|\src{})$ of translating
%a source sentence, $\src{1},\ldots,\src{n}$, to a target sentence, $\tgt{1},\ldots,\tgt{m}$.
%It accomplishes this goal through an {\it encoder-decoder} framework
%\cite{kal13,sutskever2014sequence,cho14}. The {\it encoder} computes a representation $\MB{s}$
%for each source sentence. Based on that source representation,
%the {\it decoder} generates a translation, one target word at a time, and hence,
%decomposes the log conditional probability as:
%\begin{equation}
%\log p(\tgt{}|\src{}) = \sum_{t=1}^m \nolimits \log
%p\open{\tgt{t}|\tgt{<t},\MB{s}}
%\label{e:s2s}
%\end{equation}
%
%Most NMT work uses RNNs, but approaches differ in terms of: 
%(a) architecture, which can be unidirectional, bidirectional, or deep multi-layer RNN; 
%and (b) RNN type, which can be Long Short-Term Memory (LSTM) \cite{hochreiter1997long} or the Gated Recurrent Unit \cite{cho14}. 

In this work, we specifically consider the {\it deep multi-layer recurrent} architecture with {\it
LSTM} as the hidden unit type.
Figure \ref{fig:nmt_simple} illustrates an instance of that architecture during training in which the source and target sentence pair are input for supervised
learning. During testing, the target sentence is not known in advance; instead, the most probable
target words predicted by the model are fed as inputs into the next timestep.
The network stops when it emits the end-of-sentence symbol --- a special `word' in the vocabulary, represented by a dash in Figure \ref{fig:nmt_simple}.


\subsubsection{Understanding NMT Weights}
\label{subsubsec:lstm}
Figure~\ref{fig:nmt_complex} shows the same system in more detail,
highlighting the different types of parameters, or weights, in the model.
We will go through the architecture from bottom to top.
First, a vocabulary is chosen for each language, assuming that the top $V$ frequent
words are selected.
Thus, every word in the source or target vocabulary can be represented by a one-hot vector of length $V$.
The source input sentence and target input sentence, represented as a sequence
of one-hot vectors, are transformed into a sequence of word embeddings by the
\emph{embedding} weights. 
These embedding weights, which are learned during training, are different for the source words and the target words.
The word embeddings and all hidden layers are vectors of length $n$ (a chosen hyperparameter).

The word embeddings are then fed as input into the main network, which consists
of two multi-layer RNNs `stuck together' --- an encoder for the source
language and a decoder for the target language, each with their own
weights. 
The \emph{feed-forward} (vertical) weights connect
the hidden unit from the layer below to the upper RNN block, and the
\emph{recurrent} (horizontal) weights connect the hidden unit from the previous
time-step RNN block to the current time-step RNN block.

The hidden state at the top layer of the decoder is fed through an
\textit{attention} layer, which guides the translation by `paying attention' to relevant parts of the source sentence; 
for more information see \cite{bahdanau2014neural} or Section 3 of \cite{luong2015effective}.
Finally, for each target word, the top layer hidden unit is transformed by the
\emph{softmax} weights into a score vector of length $V$. The target word with the highest score is selected as the output translation.

\bi{Weight Subgroups in LSTM} -- For the aforementioned RNN block, we choose to
use LSTM as the hidden unit type. To facilitate our later discussion 
on the different subgroups of weights
within LSTM, we first review the details of LSTM as formulated by 
\newcite{zaremba2014recurrent} as follows:
\begin{align}
\begin{pmatrix}
i\\
f\\
o\\
\hat{h}
\end{pmatrix}
&=
\begin{pmatrix}
\text{sigm}\\
\text{sigm}\\
\text{sigm}\\
\text{tanh}
\end{pmatrix}
T_{4n,2n}
\begin{pmatrix}
h_t^{l-1}\\
h_{t-1}^l
\end{pmatrix} \label{eqn:lstm_1} \\
c_t^l&=f \circ c_{t-1}^l + i \circ \hat{h} \label{eqn:lstm_2} \\
h_t^l &= o \circ \text{tanh}(c_t^l) \label{eqn:lstm_3}
\end{align}
Here, each LSTM block at time $t$ and layer $l$ computes as output a pair of
hidden and memory vectors ($h_t^l$, $c_t^l$) given the previous pair
($h_{t-1}^l$, $c_{t-1}^l$) and an input vector $h_t^{l-1}$ (either from the LSTM block below or
the embedding weights if $l\!=\!1$). All of these vectors
have length $n$.

The core of a LSTM block is the weight matrix $T_{4n,2n}$ of size $4n \times
2n$. This matrix can be decomposed into 8 subgroups that are responsible for the
interactions between $\{$input gate $i$, forget gate $f$, output gate $o$,
input signal $\hat{h}\} \times \{$feed-forward input $h_t^{l-1}$, recurrent
input $h_{t-1}^l\}$.

%\subsection{Evaluation metrics}
%We evaluate our models by two measures: BLEU score and perplexity. 
%BLEU compares the output target sentence with the human-translated target sentence, and is measured on a scale from 0 (worst) to 100 (best).
%Perplexity is the exponential of the average loss per word, measured on a scale from 1 (best) to infinity (worst). 
%%For each output target word, the model produces scores for each word in the vocabulary, which are converted to a probability distribution over the vocabulary. 
%The \emph{loss} is the negative log probability of the correct word --- that is, the lower the system's certainty in choosing the correct word, the higher the loss.
%
%Both evaluation metrics are valuable. 
%BLEU measures a system's performance on the end-goal of machine translation, translation quality, whereas perplexity is the quantity minimized during training.
%BLEU is a `hard' measure of performance, as it is calculated based on the sentences produced by the network, whereas perplexity is a `softer', more fine-grained measure that takes into account not just whether the correct target word was produced, but the probability of producing the correct target word.
%We use both BLEU and perplexity in this paper, as appropriate.

\subsubsection{Pruning Schemes}
\label{subsubsec:approach_schemes}
We follow the general magnitude-based approach of \cite{han2015learning}, which consists of pruning weights with smallest absolute value. However, we question the authors' pruning scheme with respect to the different weight classes, and experiment with three pruning schemes.
Suppose we wish to prune $x$\% of the total parameters in the model. 
How do we distribute the pruning over the different weight classes (illustrated in Figure~\ref{fig:nmt_complex}) of our model? 
We propose to examine three different pruning schemes:
\begin{enumerate}
\item \textit{Class-blind}: 
Take all parameters, sort them by magnitude and prune the $x$\% with smallest magnitude, regardless of weight class.
(So some classes are pruned proportionally more than others).
\item \textit{Class-uniform}: 
Within each class, sort the weights by magnitude and prune the $x$\% with smallest magnitude.
(So all classes have exactly $x$\% of their parameters pruned).
\item \textit{Class-distribution}: 
 For each class $c$, weights with magnitude less than $\lambda \sigma_c$ are
 pruned. Here, $\sigma_c$ is the standard deviation of that class and $\lambda$ is a universal parameter chosen such that in total, $x\%$ of all parameters are pruned.
This is used by \cite{han2015learning}.
\end{enumerate}
All these schemes have their seeming advantages.
Class-blind pruning is the simplest and adheres to the principle that pruning
weights (or equivalently, setting them to zero) is least damaging when
those weights are small, regardless of their locations in the architecture.
Class-uniform pruning and class-distribution pruning both seek to prune
proportionally within each weight class, either absolutely, or relative to the
standard deviation of that class.
We find that class-blind pruning outperforms both other schemes (see Section~\ref{subsec:exp_schemes}).

\subsubsection{Retraining}
\label{subsubsec:approach_retraining}
In order to prune NMT models aggressively without performance loss, we retrain our pruned networks. 
That is, we continue to train the remaining weights, but maintain the sparse structure introduced by pruning.
In our implementation, pruned weights are represented by zeros in the weight matrices, 
and we use binary `mask' matrices, which represent the sparse structure of a network, 
to ignore updates to weights at pruned locations.
This implementation has the advantage of simplicity as it requires minimal changes to the training and deployment code, 
but we note that a more complex implementation utilizing sparse matrices and sparse matrix multiplication could potentially yield speed improvements.
However, such an implementation is beyond the scope of this paper.


