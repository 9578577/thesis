
Neural Machine
Translation (NMT) is a novel approach to MT that has 
achieved promising results \cite{kal13,sutskever14,cho14,bog15,jean15}. 
An NMT system is a conceptually simple large neural network that 
reads the entire source sentence and produces an output translation one word at a time.
NMT systems are appealing because they use minimal domain knowledge which
makes them well-suited to any problem that can be formulated as mapping an input sequence 
to an output sequence \cite{sutskever14}.
In addition, the natural ability of neural networks to generalize implies that 
NMT systems will also generalize to novel word phrases and sentences that do not occur in the
training set. 
In addition, NMT systems potentially remove the need to store explicit phrase tables 
and language models which are used in conventional systems. 
Finally, the decoder of an NMT system is easy to implement, unlike the highly
intricate decoders used by phrase-based systems \cite{Koehn:2003:SMT}.

Despite these advantages, conventional NMT systems are incapable of translating rare 
words because they have a fixed modest-sized vocabulary\footnote{ Due to the computationally intensive nature of the softmax, NMT systems often limit 
their vocabularies to be the top 30K-80K most frequent words in each language. However, \newcite{jean15}
has very recently proposed an efficient approximation to the softmax that allows
for training NTMs with very large vocabularies. As discussed in Section~\ref{sec:nmt}, this technique is complementary to ours.}
%, although decoding time grows linearly with the size of the
%vocabulary. {\bf ILYA: their paper used a candidate list to speed up decoding.}}
which forces them to use the \unksym{} symbol to 
represent the large number of out-of-vocabulary (OOV) words, as illustrated in Figure~\ref{f:sent_pair}.
Unsurprisingly, both \newcite{sutskever14} and \newcite{bog15} have
observed that sentences with many rare words tend to be translated much more poorly than sentences
containing mainly frequent words.
%
Standard phrase-based systems \cite{koehn2007moses,chiang07hiero,cer10phrasal,dyer10cdec}, 
on the other hand, do not suffer from the rare word 
problem to the same extent because they can support a much larger vocabulary, 
and because their use of explicit alignments
and phrase tables allows them to memorize the translations 
of even extremely rare words. 

Motivated by the strengths of standard phrase-based system, we
propose and implement a novel approach to address the rare word problem of NMTs.
Our approach annotates the training corpus with 
explicit alignment information that enables the NMT system to emit, for each OOV word, a
``pointer'' to its corresponding word in the source sentence. This
information is later utilized in a post-processing step that translates
the OOV words using a dictionary or with the identity translation, if no translation is found.

Our experiments confirm that this approach is effective. On the English to French WMT'14
translation task, this approach provides an improvement of
up to \bestunkimp{} (if the vocabulary is relatively small) 
BLEU points over an equivalent NMT system that does not use this technique.
Moreover, our system is the first NMT that outperforms the winner of a WMT'14 task.

\begin{figure*}[tbh!]
\resizebox{10cm}{!}{
\setlength{\unitlength}{1.1cm}
\begin{picture}(10, 2.7) %(-3,-2)
\put(0,2){{\it en}: The \unkword{ecotax} portico in \unkword{Pont-de-Buis} , \ldots [truncated] \ldots , was taken down on Thursday morning}
\put(0,1){{\it fr}: \mbox{} Le \unkword{portique} \unkword{\'{e}cotaxe} de \unkword{Pont-de-Buis} , \ldots [truncated] \ldots , a \'{e}t\'{e} \unkword{d\'{e}mont\'{e}} jeudi matin}
\put(0,0){{\it nn}: Le \unksym{} de \unksym{} \`{a} \unksym{} , \ldots [truncated] \ldots , a \'{e}t\'{e} pris le jeudi matin}
\put(1.7,1.3){\line(2,1){1.2}} % portico
\put(3.0,1.3){\line(-2,1){1.2}} % ecotax 
\put(5.2,1.3){\line(-1,2){0.3}} % Pont-de-Buis
\put(10.8,1.3){\line(-1,1){0.6}} % taken
\put(10.8,1.3){\line(1,3){0.2}} % down
\put(12,1.3){\line(3,2){0.9}} % Thursday
\put(13,1.3){\line(2,1){1.2}} % morning
\end{picture}
}
\caption[Example of the rare word problem]{{\bf Example of the rare word problem} -- An English source sentence ({\it en}), a human translation to French ({\it fr}), and a translation produced by one of our neural network systems ({\it nn}) before handling OOV words. We highlight \unkword{words} that are unknown to our model. 
The token \unksym{} indicates an OOV word. 
We also show a few important alignments between the pair of sentences. 
}
\label{f:sent_pair}
\end{figure*}


