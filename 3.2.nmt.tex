
A neural machine translation system is any neural network that maps a source 
sentence, $s_1,\ldots,s_n$,
to a target sentence, $t_1,\ldots,t_m$, where all sentences are assumed to 
terminate with a special
``end-of-sentence'' token \eossym{}.  More concretely, an NMT system uses a neural 
network to parameterize the conditional distributions
\begin{equation}
p(t_j|t_{<j},s_{\leq n})
\end{equation}
for $1\leq j \leq m$.  By doing so, it becomes possible to 
compute and therefore maximize the log probability
of the target sentence given the source sentence
\begin{equation}
\log p(t|s) = \sum_{j=1}^m  \log p\left(t_j|t_{<j},s_{\leq n}\right)
\end{equation}
There are many ways to parameterize these conditional distributions.
For example, \newcite{kal13} used a combination of a
convolutional neural network and a recurrent neural network, \newcite{sutskever14} used a deep Long Short-Term Memory
(LSTM) model, \newcite{cho14} used an architecture similar to the LSTM, and
\newcite{bog15} used a more elaborate neural network
architecture that uses an attentional mechanism over the input sequence, 
similar to \newcite{graves13c} and \newcite{graves14}.  

In this work, we use the model of \newcite{sutskever14}, which 
uses a deep LSTM to encode the input sequence and a separate deep LSTM 
to output the translation. The encoder reads the 
source sentence, one word at a time, and produces
a large vector that represents the entire source sentence. 
The decoder is initialized with this vector
and generates a translation, one word at a time, 
until it emits the end-of-sentence symbol \eossym{}.

None the early work in neural machine translation systems has addressed the rare word problem,
but the recent work of \newcite{jean15} has tackled it with 
an efficient approximation to the softmax to accommodate for a very large vocabulary (500K words). However, even with a large vocabulary, the problem with rare words, e.g., names, numbers, etc., still persists, and \newcite{jean15} found that using techniques similar to ours are beneficial and complementary to their approach.
%, and a combination
%of both methods achieves even greater performance {\bf ILYA: they tried UNK replacement but session 3.3 in their paper never cites us (they mentioned us in Section 2.2), and yeah they obtained gains from doing so. When you said ``even greater performance'', it may sound like they even obtained better performance than us?}. 
