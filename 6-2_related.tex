Pruning the parameters from a neural network, referred to as \textit{weight pruning} or \textit{network pruning}, is a well-established idea though it can be implemented in many ways. 
Among the most popular are the Optimal Brain Damage (OBD)
\cite{lecun1989optimal} and Optimal Brain Surgeon (OBS) \cite{hassibi1993second} techniques, which involve computing the Hessian matrix of the loss function with respect to the parameters, in order to assess the \textit{saliency} of each parameter. 
Parameters with low saliency are then pruned from the network and the remaining sparse network is retrained. 
Both OBD and OBS were shown to perform better than the so-called `naive magnitude-based approach', which prunes parameters according to their magnitude (deleting parameters close to zero).
However, the high computational complexity of OBD and OBS compare unfavorably to the computational simplicity of the magnitude-based approach, especially for large networks \cite{augasta2013pruning}.

In recent years, the deep learning renaissance has prompted a re-investigation of network pruning for modern models and tasks. 
Magnitude-based pruning with iterative retraining has yielded strong results for Convolutional Neural Networks (CNN) performing visual tasks.
\cite{collins2014memory} prune 75\% of AlexNet parameters with small accuracy loss on the ImageNet task, while \cite{han2015learning} prune 89\% of AlexNet parameters with no accuracy loss on the ImageNet task.

Other approaches focus on pruning neurons rather than parameters, via sparsity-inducing regularizers \cite{murray2015auto} or `wiring together' pairs of neurons with similar input weights \cite{srinivas2015data}. These approaches are much more constrained than weight-pruning schemes; they necessitate finding entire zero rows of weight matrices, or near-identical pairs of rows, in order to prune a single neuron. By contrast weight-pruning approaches allow weights to be pruned freely and independently of each other. The neuron-pruning approach of \cite{srinivas2015data} was shown to perform poorly (it suffered performance loss after removing only 35\% of AlexNet parameters) compared to the weight-pruning approach of \cite{han2015learning}. 
Though \cite{murray2015auto} demonstrates neuron-pruning for language modeling as part of a (non-neural) Machine Translation pipeline, their approach is more geared towards architecture selection than compression.

\begin{sloppypar}
There are many other compression techniques for neural networks, including approaches based on on low-rank approximations for weight matrices \cite{jaderberg2014speeding,denton2014exploiting}, or weight sharing  via hash functions \cite{chen2015compressing}.
Several methods involve reducing the precision of the weights or activations \cite{courbariaux2015low}, sometimes in conjunction with specialized hardware \cite{gupta2015deep}, or even using binary weights \cite{lin2015neural}.
The `knowledge distillation' technique of \cite{hinton2015distilling} involves training a small `student' network on the soft outputs of a large `teacher' network.
Some approaches use a sophisticated pipeline of several techniques to achieve impressive feats of compression \cite{han2015deep,iandola2016squeezenet}.
\end{sloppypar}

Most of the above work has focused on compressing CNNs for vision tasks. 
I extend the magnitude-based pruning approach of \cite{han2015learning} to recurrent neural networks (RNN), in particular LSTM architectures for NMT, and to my knowledge I am the first to do so.
There has been some recent work on compression for RNNs \cite{lu2016learning,prabhavalkar2016compression}, but it focuses on other, non-pruning compression techniques. 
Nonetheless, my general observations on the distribution of redundancy in a
LSTM, detailed in Section \ref{subsubsec:redundancy}, are corroborated by \cite{lu2016learning}.
